\noindent The work of this thesis is a combination of author's own contribution and the contribution of others. The works related to the training, testing and verifying classifier performance described in the section [???] are entirely the author's work. 

\noindent The development and maintenance of TbUT software, described in section [???], is entirely the author's contribution. 

\noindent Significant contribution were made by the author to the testbeam analysis, in particular the entire section on charge sharing analysis were author's work, except where referenced. Author took part in the shifts during a multiple of testbeam campaigns at CERN, where a lot of the data used in the thesis came from. 

The list of papers with my contribution, conferences and summer schools that I participated is provided below. 

\section*{Publications}
			\begin{itemize}
			    \item \textbf{Machine learning based long-lived particle reconstruction algorithm for the LHCb experiment},\\ NeuralPS workshop "Machine Learning and the Physical Sciences"; (\textcolor{red}{\href{https://ml4physicalsciences.github.io/2020/files/NeurIPS_ML4PS_2020_148.pdf}{paper}},\textcolor{red}{\href{https://ml4physicalsciences.github.io/2020/files/NeurIPS_ML4PS_2020_148_poster.pdf}{poster}});
			    \item \textbf{PatLongLivedTracking: a tracking algorithm for the reconstruction of the daughters of long-lived particles in LHCb};\\ LHCb-PUB-2017-001;
				\item \textbf{Emulation and Calibration of the SALT Read-out Chip for the Upstream Tracker for Modernised LHCb Detector}; \\
				Acta Phys. Pol. B 46 (2015) 1263-1269;
				\item \textbf{Testbeam studies of pre-prototype silicon strip sensors for the LHCb UT upgrade project};  \\
				Nucl.Instrum.Meth. A806 (2016) 244-257;
				\item \textbf{Signal coupling to embedded pitch adapters in silicon sensors};\\
				Nucl.Instrum.Meth. A877 (2018) 252-258.
			\end{itemize}

%---------------------------------------------------------
\section*{Conferences}
\begin{itemize}
                \item \textbf{ML in PL Conference 2019 } Machine learning in High Energy Physics (\textcolor{red}{\href{https://www.youtube.com/watch?v=6cO2OBhJlDQ\&list=PLoaWrlj9TDhPf08oDhBspvSP11E_uXSnB&index=39}{recording}});
                \item \textbf{76th LHCb Analysis and Software week} Building and validating MVAs, How to build more reliable ML models (\textcolor{red}{\href{https://docs.google.com/presentation/d/14ZELcsYN_eMCpxtsYzOmIaC_o8rhU6abqH1jwxCijZI/edit?usp=sharing}{slides}});
                
			    \item \textbf{8th International Conference on New Frontiers in Physics (ICNFP 2019)}; Machine Learning techniques used in LHCb analyses and online applications;
			    \item  \textbf{LHCP, Bologna, 4-9 June 2018} A tracking algorithm for the reconstruction of the daughters of long-lived particles in LHCb;
				\item \textbf{Connecting the Dots/Inteligent Tracker; 2017;} Deep Neural Nets and Bonsai BDTs in the LHCb pattern recognition;
				\item \textbf{XXII Cracow Epiphany Conference on Physics in LHC Run II; 2016;} Calibration and monitoring of the SALT readout ASIC for the LHCb UT detector;
				\item \textbf{XXI Cracow Epiphany Conference on future of High Energy Collider; 2015;} Emulation and calibration of the SALT readout chip for the UT tracker for modernised LHCb detector.
			\end{itemize}
			
\section*{Summer schools}
\begin{itemize}
    \item \textbf{Wolfram Summer School}, July 2017, Waltham MA, USA, 
    \begin{itemize}
        \item Worked on project DeepLaetitia: Deep Reinforcement Learning That Makes You Smile (\textcolor{red}{\href{https://education.wolfram.com/summer/school/alumni/2017/dendek/}{project summary}});
    \end{itemize}
    \item \textbf{Second Machine Learning in High Energy Physic Summer School 2016}, July 2016, Lund Sweden 
    \item \textbf{The 3rd Asia-Europe-Pacific School of High Energy Physics}, October 2016, Beijing China
    \item \textbf{The 38th CERN School of Computing}, September 2015, Kavala Greece
\end{itemize}