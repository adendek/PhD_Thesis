If somebody who is about to submit an application for a Ph.D. position asked me what I think about this idea, I would tell him/her that a Ph.D. study can be compared to a marathon race.  In my case, it took six years to complete it, and it wasn’t painless. I had a lot of moments where I was within an inch of resigning. However, I reached the moment when my thesis is done, and I am waiting for reviews. Therefore, I would like to say “thank you” to some people who helped me be where I am.
 
First things first, I would like to thank my supervisor, prof. T. Szumlak. You gave me an opportunity to work on a project that involved applying machine learning techniques in a complicated scientistic scenario that forced me to study them. This knowledge will drive my future career. Secondly, I would like to thank You for allowing me to be your teaching assistant (AGH course “Python in the Enterprise”). As Richard Feynman once said, “If you want to master something, teach it,” and for me, teaching and supervising a team of undergraduate students was a very productive time. According to the surveys, most of the students enjoyed this course and thought of it as one that helped them to find a better job. Finally, I would like to thank You for the review of this document. 


Secondly, I wish to thank all members of the Upstream Tracker Testbeam team. The experience of testbeams at CERN was one that shaped me as a researcher. 
Special thanks go to prof. Steven Blusk. You are a great researcher, and you showed me a proper way how to approach scientific problems. Our cooperation, even though not successful as it should have been, taught me an unforgivable lesson. Moreover, I would like to thank Adam Davis. You supported me for so long, and you were kind and patient in answering my silly question. And finally, I would like to mention Constantin “Stan” Weisser. I learn from you one vital skill - to be bold and not being afraid of asking questions. 


These acknowledgments would not be complete without a special thanks to my old-time best friends from high school (sorted in alphabetically ascending order using the first name as a key) Bartek, Dawid, Karol, Konrad, Szymon, Tomasz, Wiktor. I hope that we stay together regardless of the distance that may separate us.

I also wish to thank Łukasz Fulek for many discussions about physics, programming, and life in general. 

This is eventually time to thank my fiancée Kasia, for sharing your life with me  and for making each every minute we spend together a very special moment.  

Ostatni paragraf w tej sekcji podziękowań chciałbym dedykować mojej Mamie. Zdaję sobię sprawę, że nie jestem w stanie słowmi określić swojej wdzięczności za wszystko, co od Ciebie otrzymałem. Pomimo, że dzielą nas duże odległości jesteś zawsze przy mnie, kiedy potrzebuję pomocy albo porady. W szczególności jestem wdzięczny za danie mi możliwości podejmowania oraz ponoszenia konsekwencji samodzielnych wyborów.  
