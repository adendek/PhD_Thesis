% the abstract

Niniejsza rozprawa doktorska składa się z opisu dwóch projektów badawczych zrealizowanych w ramach współpracy LHCb. Pierwszy z nich jest dedykowana opisowi prac nad algorytmem do rekonstrukcji śladów pochodzących od cząstek długożyciowych w ramach eksperymentu LHCb. W ramach prowadzonych badań zdecydowano się zastosować nowatorskie metody uczenia maszynowego w celu poprawy czystości i wydajności rekonstrukcji. Projekt ten jest jednym z pierwszych, który używa zaawansowanych modeli uczenia maszynowego w ramach systemu wyzwalania wysokiego poziomu. W ramach studiów nad analizą wydajności testowanych modeli wykonano nowatorską analizę interpretowalności predykcji złożonych modeli uczenia maszynowego. 

 Druga część pracy przedstawia zaprojektowanie i zaimplementowanie platformy do emulacji i monitoringu algorytmów przetwarzania surowych danych zbieranych przez projektowany detektor UT (ang. Upstream Tracker) w ramach modernizacji detektora LHCb. W wyniku tych prac została dostarczona aplikacja TbUT. Aplikacja ta była wykorzystywana podczas szeregu testów na wiązce, których celem było sprawdzenie poprawności projektowanych sensorów oraz elektronicznego układu odczytu front-end Salt. W przyszłości oprogramowanie to będzie wykorzystywane między innymi do wykonywania kalibracji i monitorowania poprawności działania detektora UT.  


Rozprawa doktorska rozpoczyna się od wstępu, który skupia się przedstawieniu eksperymentu LHCb oraz wyjaśnieniu zasady działania każdego z elementów detektora, jak również motywacji do jego modernizacji. W kolejnym rozdziale zostały przedstawione aspekty teoretyczne dotyczące Modelu Standardowego ze szczególnym uwzględnieniem oddziaływań słabych oraz problemu łamania symetrii kombinowanej $CP$, będącej motywacją do powstania eksperymentu LHCb. Rozdział trzeci skupia się na przedstawieniu i dogłębnej analizie algorytmów uczenia maszynowego. Dyskutowane są zarówno matematyczne podstawy wybranych modeli, jak również procesu ich trenowania oraz optymalizacji ich hiper parametrów. Czwarty rozdział jest dedykowany przedstawieniu prac w ramach poprawy algorytmu rekonstrukcji śladów cząstek długożyciowych. Rozdział ten składa się z przedstawienia algorytmu rozpoznawania wzorców oraz studiów nad dwoma klasyfikatorami opartymi o algorytmy uczenia maszynowego. Rozdział  piąty przedstawia oprogramowanie TbUT. Szósty rozdział przedstawia analizę danych zebranych podczas testów na wiązce w szczególności skupiając się na problemie współdzielenia ładunku. Rozprawa kończy się  podsumowaniem i wnioskami zebranymi w rozdziale siódmym. 
