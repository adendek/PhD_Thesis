\begin{savequote}[75mm]
Not only is the Universe stranger than we think, it is stranger than we can think
\qauthor{Werner Heisenberg}
\end{savequote}

\chapter{Physics behind the LHCb experiment }
\label{chapter:physics}

This chapter is dedicated to providing a brief introduction to the physics of LHCb. It starts by presenting the fundamental concept of symmetries in physics, including the particular type of discrete symmetries and its consequence, then the Standard Model of particle physics is briefly described. 

\section{Symmetries in physics}
Until the 20th-century, principles of symmetry played a minor role in theoretical physics. The ancient Greeks were fascinated by the symmetries of objects and believed that these should be mirrored in the structure of nature. Still, they did not manage to associate those symmetries with any deterministic law of physics. Instead, symmetry was a critical component that inspires several architects designing the most stunning and recognizable buildings, and even recently, psychologies proved that symmetrical faces are more attractive \cite{faces}. 
In one of the most important book of all time "Philosophiæ Naturalis Principia Mathematica" \cite{newton} 
Newton postulated that laws of mechanics incorporate symmetry principles, notably the principle of equivalence of inertial frames, or Galilean invariance. These symmetries implied conservation laws. However, these conservation laws were seen as consequences of the dynamical laws of nature rather than as consequences of the underlaying symmetries.  

This situation had dramatically changed at the beginning of the 20th century when Emmy Noether proved her famous theorem relating continuous symmetries and conservation laws.
This theorem states that as a direct consequence of the invariance of physics laws under continuous spatial transformations, such as spatial translation, spatial rotation and time translation, momentum angular momentum and energy are conserved respectively. The less intuitive case is the symmetry of the wave function under the change of its phase that leads to the conservation of the electric charge. Precisely speaking, asymmetry is a mathematical operation that leaves the physical system invariant. 
From the moment of this paper publication onwards, physics can be defined as the science that studies fundamental symmetries and the mechanism of the symmetries breaking.  

In Quantum Mechanics, symmetries are mostly discrete and satisfied if an operator commutes with the Hamiltonian (representing the time transformations). If a discrete symmetry holds, it leads to a conserved quantum number and a selection rule for the transitions between different states. 
Within the framework of particle physics, there are three fundamental symmetries $P$, $C$, and $T$; those symmetries are shown in figure \ref{fig:CPT}

A parity transformation $P$ can be represented by inversion of the spatial coordinates of a coordinate system and changing the helicity of the particle. Namely, the eigenfunction of the parity operator $  \hat{P}$ satisfy the condition: 

\begin{equation}
    \hat{P}~ \ket{\Psi, \vec{r}} = \ket{\Psi, - \vec{r}} = p \ket{\Psi, \vec{r}}  
\end{equation}
Where $p$ is an eigenvalue of the $ \hat{P}$ operator and $\ket{\Psi}$ its eigenstate. A second application of the $\hat{P}$ transforms the $\ket{\Psi}$ to its initial state, therefore the $p$ must be equal to $\pm 1$, and by convention for fermions $p=1$.  


Time parity $T$ reverse the time direction of a process, and the charge conjugation $C$ changes the sign of the additive quantum numbers, transforming particles into antiparticles keeping their helicity. The eigenvalues of the  charge conjugation operator $\hat{C}$ can be expressed in the following form:
 

\begin{equation}
    \hat{C}~ \ket{particle} = \ket{anti~ particle} = c \ket{particle}  
\end{equation}

Where $c$ is an eigenvalue of the $ \hat{C}$ operator, and similarly to the $\hat{P}$ operator's eigenvalues can have one of $\pm 1$ value. It is interesting to note that only particles that
are their own antiparticles can be eigenstates of the charge parity operator.

The strong and electromagnetic interactions are invariant under each of $C$, $P$, and $T$ symmetries.  The combination of $CPT$ is an exact symmetry of any interaction described by the Lorenz invariant quantum field theory. This phenomenon is described as the $CPT$ theorem \cite{CPT_theorm} (thus far we did not find any experimental results that would negate this fundamental theorem).
It is observed that $C$ and $P$ are not exact symmetries and they are both broken by the weak interactions: the weak charged currents couple exclusively to left-handed fermions and right-handed anti-fermions, and hence maximally violate $C$ and $P$ symmetries individually. The combined $CP$ operation transforms a left-handed fermion into a right-handed anti-fermion. Therefore it could be expected that the $CP$ symmetry is the one that is held by the weak interactions. This statement was valid until 1964 when the group , led by James Cronin and Val Fitch, working on neutral kaons decays discovered that this is not a case, \cite{CPV_kaons}. 

\begin{figure}
\centering
\includegraphics[scale=1.0]{figures/CPT.png}
\caption{Symmetries in particle physics. Each arrow represents one of the $C$, $P$, $T$ transformations or their combinations. With the discovery that the weak interactions maximally violate both charge and spatial parity symmetries, it was postulated the true symmetry between matter and antimatter is the combined charge and spatial parity transformation. As it turned out, this one is also broken by the weak interaction, but in that case, the violation effects are relatively small. The combined $CPT$ transformation is treated as the fundamental property of our Universe and should hold for all processes. Figure taken from \cite{CPT}
\label{fig:CPT}}
\end{figure}

\subsection{Group Theory}
This subsection is dedicated to providing a brief introduction to Group Theory, which is a branch of mathematics that was developed to studying symmetries. This section reviews some of the properties of them.

A Group $G$ is an abstract set of elements, which can be finite or infinite, with defined operator $(\cdot)$ on it, which obeys:

\begin{itemize}
    \item Closure: $\forall u,v \in G, u\cdot v \in G$
    \item Associativity: $\forall u,v,w \in G u \cdot (v \cdot w ) = (u \cdot v) \cdot w$
    \item Neutral element:  $\exists I \in G, u \cdot I = I \cdot u = u, \forall u \in G$
    \item Inverse element:  $\exists u \in G, \exists u^{-1} \in G, u \cdot u^{-1} = I$
\end{itemize}

Group G is called Abelian or commutative if also the following axiom is satisfied:
\begin{itemize}
    \item Commutativity $\forall u,v \in G u \cdot v = v \cdot u$
\end{itemize}

In elementary particle physics, the most common groups are of the type of $U(n)$, which can be represented as a collection of all unitary $n\times n$ matrices \footnote{A unitary matrix is one whose inverse is equal to its transpose conjugate $U^{-1}=U^\dagger$}. The second type of group used to construct the Standard Model is a Super Unitary group $SU(n)$. One of the fundamental properties of the unitary matrices is that their determinant is equal to 1, this plays a vital role in quantum theory that requires probability
conservation. 
A group of matrices can represent any group, thus for every abstract element $u$, there is a corresponding matrix $M_u$, which needs to fulfil all axioms listed above. 

As a concrete example let consider a group of real, orthogonal \footnote{An orthogonal matrix is a matrix whose inverse is equal to its transpose: $O^{-1} = O^{T}$} $n\times n$ matrices with the determinant equal to  1, this group called $SO(n)$ and can represent all rotation is space of $n$ dimensions. For $n=3$, group $SO(3)$  describes rotational symmetry of our word, which according to the Noether's theorem is equivalent to the conservation of angular momentum \cite{griffiths}. 

\section{Standard Model of elementary particles}
\label{sec:SM}

The Standard Model (SM) is a relativistic Quantum Field Theory that describes properties of elementary particles and the electromagnetic, weak, and strong interactions. It has been experimentally validated on numerous occasions,
making very accurate predictions. However, it is an incomplete theory since it does not include the theory of gravity and does not explain the dark matter and dark energy, the neutrino masses, nor the matter-antimatter asymmetry of the Universe.

This theory was built to model the particle interactions observed in nature. Using this framework, one can obtain predictions for physical phenomena by calculating transition probability from an initial state $\bra{i, i'}$ to a given final state$\ket{k, k' }$. These calculations are performed using Quantum Field Theory tools such as expansions of a path integral into a power series, which can be visualized using Feynman diagrams (one diagram per term). Each term in those series can be interpreted as a particular interaction process. 

According to the Standard Model, the whole space is filled with different types of fields, and the excitation of those fields are what can be interpreted as particles, the visualization of properties of elementary particles are shown in figure \ref{fig:SM} . The matter is built by twelve particles called fermions \footnote{And their corresponding antiparticles}, which has a half-integer spin, and they have to obey the Pauli exclusion principle. The interactions between fermions are perceived as an exchange of integer spin particles called bosons. 
None of them is known to have any underlying substructure. Thus they are called fundamental particles.
All fermions can be further split into two groups: quarks and leptons. This distinction is driven by the interaction in which a particular fermion can participate. 


The mathematical structure of the Standard Model can be described by the following symmetry group:

\begin{equation}
    SU(3)_C \times SU(2)_L \times U(1)_Y
\end{equation}

The group $SU(3)_C$ represents symmetry transformations in an internal colour space that are used to describe properties of the strong interactions. The dimension of the group corresponds to three types of colour charge. The second $SU(2)_L \times U(1)_Y$ symmetry group describes electro-weak interactions and takes into account the spontaneous symmetry breaking leading to massive intermediate bosons. The $SU(2)_L$ group describes rotations in an abstract weak-isospin space and the subscript $L$ is used to express the fact that the fundamental representations of the weak interactions are left-handed dublets and right-handed singlets. The remaining $U(1)_Y$ group describes the electromagnetic interactions that should conserve weak hypercharge $Y$. In this case, the symmetry transformations can be interpreted as phase shifts of the wave function. The core idea behind the significance of this combined symmetry group is that the Lagrangian describing any process occurring in Nature must be invariant with respect to any transformation that belongs to it.

The strong interactions, which are mediated by eight massless gluons carrying a quantum number called colour and occurs between quarks, those interactions are described by a theory called Quantum Chromodynamic (QCT). Among all fermions, only quarks can carry colour charge, which allows them to interact via the strong interaction. There are six types of quarks known as flavors: up $u$,  down $d$, charm $c$, strange $s$, top $t$ and beauty $b$.  Due to the unique nature of the strong interaction, in which the mediators of the force, the gluons, carry the same colour charge that they mediate, quarks are \footnote{This happens for all quarks besides the top quark, which lifetime is too short to combine with other quarks to form hadrons} "glued" to quarks and in Nature they always form colourless composite particles called hadrons. 

Depending on the number of quarks they are made of, hadrons are classified as mesons (composed of $q\bar{q}$ pair), baryons (with three quarks combining all the colors) and exotic hadrons composed of four and five quarks, which has been recently discovered and reported by the LHCb collaboration \cite{pentaquarks}.  

\begin{figure}
\centering
\includegraphics[scale=1.0]{figures/SM.png}
\caption{Illustration of the fundamental particles in the Standard Model and their properties. The fermions are organized in three generations (denoted by I, II, and III) comprised of quarks preseted as the purple squares and leptons as the green squares. The  gauge bosons are shown as red circles, and the yellow square indicates the scalar Higgs boson 
\label{fig:SM}}
\end{figure}

\section{Weak interactions and CKM matrix}

As explained in the previous section, the $ SU(2)_L \times U(1)_Y$ gauge group describes the properties of the electro-weak interactions. Those interactions are carried by four vector (i.e., spin 1) bosons, namely $W^{\pm}$, $Z$, and the $\gamma$.  The first three of them mediate the weak interactions, and $\gamma$ is responsible for carrying electromagnetic interactions. One of the key features of the weak interactions is the experimental observation that they couple to the left-handed particles only, which is a direct manifestation of the maximal violation of the charge and spatial parity symmetries. Another unique feature of the weak forces is the mixing between different quarks families, which leads in consequence to the experimental observation that the quark mass eigenstates are not the same as the weak eigenstates. The Cabibbo-Kobayashi-Maskawa (CKM) matrix relates the weak eigenstates,  ($d\prime$,$s\prime$,$b\prime$), with the mass eigenstates, ($d$,$s$, $b$), and is written as:

\begin{equation}
\label{eq:ckm}
  \begin{bmatrix}  d^\prime  \\  s^\prime  \\  b^\prime  \end{bmatrix} = \begin{bmatrix} V_{ud} & V_{us} & V_{ub} \\ V_{cd} & V_{cs} & V_{cb} \\ V_{td} & V_{ts} & V_{tb} \end{bmatrix} \begin{bmatrix}  d  \\  s  \\  b  \end{bmatrix}
\end{equation}

where the $3 \times 3$ unitary matrix $V_{CKM}$ is known as the Cabibbo-Kobayashi-Maskawa (CKM) quark mixing matrix\cite{ckm1}\cite{ckm2}. The magnitude of each element in $V_{CKM}$ represents the coupling strength of the quarks in the subscript to the weak field $W^{\pm}$.  One of the properties of such matrix is the possibility to fully describe its transformation properties by three Euler angles and one complex phase. This phase parameter is vital for the theory since its non-zero value can explain, within the theoretical framework of the SM, the violation of Charge-Parity $CP$ symmetry. Using the relation between the unitary matrices and rotation matrices the CKM matrix can be decomposed as follow:

\begin{align} V_{CKM}  &= \begin{bmatrix} 1 & 0 & 0 \\ 0 & c_{23} & s_{23} \\ 0 & -s_{23} & c_{23} \end{bmatrix}
 \begin{bmatrix} c_{13} & 0 & s_{13}e^{-i\delta_{13}} \\ 0 & 1 & 0 \\ -s_{13}e^{i\delta_{13}} & 0 & c_{13} \end{bmatrix}
 \begin{bmatrix} c_{12} & s_{12} & 0 \\ -s_{12} & c_{12} & 0 \\ 0 & 0 & 1 \end{bmatrix} \nonumber \\ 
 & = \begin{bmatrix} c_{12}c_{13} & s_{12} c_{13} & s_{13}e^{-i\delta_{13}} \\
 -s_{12}c_{23} - c_{12}s_{23}s_{13}e^{i\delta_{13}} & c_{12}c_{23} - s_{12}s_{23}s_{13}e^{i\delta_{13}} & s_{23}c_{13}\\
 s_{12}s_{23} - c_{12}c_{23}s_{13}e^{i\delta_{13}} & -c_{12}s_{23} - s_{12}c_{23}s_{13}e^{i\delta_{13}} & c_{23}c_{13} \end{bmatrix}
\end{align}
where $c_{ij} =\cos(\theta_{ij})$, $s_{ij} =\sin(\theta_{ij})$,  $\theta_{ij}$ are the respective quark mixing angles, and $\delta_{13}$ is a complex phase. Angle $\theta_{12}$ is called Cabbibo angle, which was introduced in 1963 when only two generation of quarks were known \cite{cabibbo}. 


One of the customary parametrizations of the $V_{CKM}$, that is convenient to present the hierarchical structure of its parameters is called Wolfenstein \cite{wolfenstein} parametrization:.

\begin{equation}
\label{eq:wolfenstein}
   V_{CKM} =  \begin{bmatrix} 1-\tfrac{1}{2}\lambda^2 & \lambda & A\lambda^3(\rho-i\eta) \\
 -\lambda & 1-\tfrac{1}{2}\lambda^2 & A\lambda^2 \\
 A\lambda^3(1-\rho-i\eta) & -A\lambda^2 & 1  \end{bmatrix} + O(\lambda^4).
\end{equation}

Experimentally, it is found that the mixing between mass and weak eigenstates is relatively small in the quark sector. Based on current knowledge \cite{CKMFitter}, the values of $V_{CKM}$ parameters are:
\begin{itemize}
\item $\lambda=0.224837^{+0.000251}_{-0.000060}$,
 \item  $A= 0.8235^{+0.0056}_{-0.0145]}$, 
 \item  $\rho=0.1569^{+0.0102}_{-0.0061}$,
 \item  $\eta=0.3499^{+0.0079}_{-0.0065}$.
\end{itemize}

 Therefore each of the mixing angles, represented as an off-diagonal element \ref{eq:ckm}, is small, and the CKM matrix is approximately diagonal, see figure \ref{fig:ckm_magnitudes}. Therefore, processes that involve off-diagonal elements of the CKM matrix, those that change the generation of the quarks are Cabibbo-suppressed with respect to those on the diagonal, which are referred to as Cabibbo-favoured.

\begin{figure}[h]
\centering
\includegraphics[scale=0.5]{figures/ckm_structure.PNG}
\caption{Hierarchy of the $V_{CKM}$ matrix elements. The numbers are approximate to illustrate the relative magnitudes of the elements.  
\label{fig:ckm_magnitudes}}
\end{figure}

 The $V_{CKM}$ matrix is unitary one what leads to a series of relationships between different elements that can be experimentally probed. These constraints impose the following relations:

\begin{equation}
\label{eq:unitary}
\begin{split}
        V_{1j}^{*}V_{1k} +  V_{2j}^{*}V_{2k} +  V_{3j}^{*}V_{3k} = \delta_{jk} \\
        V_{j1}^{*}V_{k1} +  V_{j2}^{*}V_{k2} +  V_{j3}^{*}V_{k3} = \delta_{jk} 
\end{split}
\end{equation}

Those six relations, summarized by equation \ref{eq:unitary}, can be visualized as, so called, unitarity triangles in the complex plane. The most popular to study is the triangle with  $j=b$ and $k=d$. It drew attention due to having all sides of the similar sizes. This triangle can be completely constructed by defining that one of the sides lies on the real axis and then by defining its apex as:
\begin{equation}
   (\overline{\rho}, \overline{\eta}) =- \frac{V_{ub}^{*}V_{ud}}{V_{cb}^{*}V_{cd}}
\end{equation}
while the remaining apexes are $(0,0)$ and $(0,1)$.  
The angles of this unitary triangle denoted as $\alpha, \beta, \gamma$ are defined as follow: 

\begin{align*}
   \alpha &= arg\left( \frac{V_{tb}^{*}V_{td}}{V_{ub}^{*}V_{ud}} \right), & 
   \beta &=  arg\left( \frac{V_{cb}^{*}V_{cd}}{V_{tb}^{*}V_{td}} \right), &
   \gamma &= arg\left( \frac{V_{ub}^{*}V_{ud}}{V_{cb}^{*}V_{cd}} \right) 
\end{align*}



\begin{figure}[h]
\centering
\includegraphics[scale=0.8]{figures/Unitary_triangle.PNG}
\caption{Visualization of one of the unitary triangles of the $V_{CKM}$ matrix. Definition of the angles $\alpha, \beta, \gamma$ can be found in text.   
\label{fig:triangle}}
\end{figure}

The values of those angles cannot be predicted using the Standard Model framework. Instead, they must be measured experimentally. Any possible discrepancy in relation \ref{eq:unitary} may indicate a contribution from physics beyond the Standard Model. Figure \ref{fig:triangle} shows the overall status of the CKM unitary triangle in the $\overline{\rho}, \overline{\eta}$ plane, with a global fit to the apex. Within the current experimental uncertainties, there are no significant deviations from the Standard Model predictions. 

\begin{figure}
\centering
\includegraphics[scale=0.7]{figures/Unitary_triangle_constrains.PNG}
\caption{Overlapping constraints of the CKM unitary triangle. The red dashed area indicates global fit of the CKM apex with 68\% confidence interval. Figure adopted from \cite{CKMFitter}.
\label{fig:triangle}}
\end{figure}


\section{Neutral Meson Mixing and $CP$ violation}

One of the most astonishing phenomena in physics is neutral meson mixing, i.e., the ability to change into its antiparticle spontaneously. Such transitions are related with the corresponding flavour quantum number violation (strangeness for $K^0$ mesons, charm for $D^0$ mesons and beauty for $B^0$, and $B^0_ s$ mesons) and they can only be induced by the weak interactions. Mixing, also called flavour oscillation, is also an essential source of $CP$ violation in the SM. Those mixing processes can be described by the so-called box diagram presented in figure \ref{fig:Mixinig}. 


\begin{figure}
\centering
\includegraphics[scale=0.9]{figures/Box-diagrams-depicting-K-0-K-0-mixing.png}
\caption{Box diagrams illustrating the mixing of $K^{0}$ mesons. In both cases virtual $W$ bosons
and quarks connect the four vertices. Figure adopted from \cite{Mixing}.
\label{fig:Mixinig}}
\end{figure}


The mass eigenstates, that propagate in space, are written as a superposition of flavor eigenstates:
\begin{equation}
    \begin{split}
        \ket{M_L} = p \ket{M} + q \ket{\overline{M}} \\
        \ket{M_H} = p \ket{M} - q \ket{\overline{M}} \\
    \end{split}
\end{equation}
 where $p$ and $q$ are complex numbers satisfying the normalization condition $|p|^{2}+|q|^{2} = 1$. 
 The evolution of this system is described by the non-hermitian Hamiltonian, given as:

\begin{equation}
\label{eq:mixing_hamiltonian}
    \mathcal{H}  = M - \frac{i}{2} \Gamma
\end{equation}

where $M$ and $\bm{\Gamma}$ are the mass and decay matrices, respectively, defined as:

\begin{align}
    M &= \left( \begin{matrix} M_{11} & M_{12}  \\ M_{12}^{*} & M_{22} \end{matrix} \right), & 
    \bm{\Gamma} &=  \left( \begin{matrix} \Gamma_{11} & \Gamma_{12}  \\ \Gamma_{12}^{*} & \Gamma_{22} \end{matrix} \right) 
\end{align}

The Hamiltonian of the process, equation \ref{eq:mixing_hamiltonian}, is not hermitian, thus ensures that the neutral meson may eventually decay. The diagonal elements of both $M$ and $\bm{\Gamma}$ matrices are the same due to $CPT$  theorem, and if the off-diagonal elements are equal to zero the meson states are degenerated and the mixing would not occur. Moreover, both  $M$ and $\bm{\Gamma}$ are hermitian operators. It is customary to define the average mass and average decay widths as follow:

\begin{equation}
\label{eq:mass_gamma}
    M \equiv \frac{M_{H} + M_{L}}{2}, ~~~~
    \Gamma \equiv \frac{\Gamma_{H} + \Gamma_{L}}{2}
\end{equation}
Next, two parameters x and y can be defined using the mass and decay width differences respectively:
\begin{equation}
\label{eq:x_y}
    x \equiv \frac{M_{H} - M_{L}}{\Gamma} = \frac{\Delta M}{\Gamma}, ~~~~
    y \equiv \frac{\Gamma_{H} - \Gamma_{L}}{2 \Gamma} = \frac{\Delta \Gamma}{2 \Gamma}
\end{equation}
The mass difference, $\Delta M$, can be related to the mixing frequency and together with the difference of decay width, $\Delta \Gamma$, can be measured experimentally.

The time evolution of the flavor eigenstates obeys the time-dependent Schr\"{o}dinger equation:

\begin{align}
\label{eq:time dependent hamiltonian}
\centering
    i\frac{d}{dt} \left(\begin{matrix} \ket{M(t)}  \\ \ket{\overline{M}(t)} \end{matrix}  \right) 
    &= \mathcal{H} \left(\begin{matrix} \ket{M(t)}  \\ \ket{\overline{M}(t)} \end{matrix}  \right) \\
    i\frac{d}{dt} \left(\begin{matrix} \ket{M(t)}  \\ \ket{\overline{M}(t)} \end{matrix}  \right) 
    &=\left( \begin{matrix} M_{11} - i\frac{1}{2}\Gamma_{11} & M_{12} -i\frac{1}{2}\Gamma_{11} \\ M_{12}^{*} - i\frac{1}{2}\Gamma_{12}^{*} & M_22-i\frac{1}{2}\Gamma_{22} \end{matrix} \right) \left(\begin{matrix} \ket{M(t)}  \\ \ket{\overline{M}(t)} \end{matrix}  \right) 
\end{align}

The solution of the equation \ref{eq:time dependent hamiltonian} is a pair of eigenvalues given by:

\begin{align}
\centering
    v_{1} &= \left( \begin{matrix} p  \\ q \end{matrix} \right), & 
    v_{2} &=  \left( \begin{matrix} p \\ -q \end{matrix} \right) 
\end{align}

Those eigenvalues allows to diagonalize the hamiltonian: 

\begin{align}
   Q\mathcal{H}Q^{-1} &= P =  \left( \begin{matrix} M_{L} - \frac{i}{2}\Gamma_{L} & 0  \\ 0 &  M_{H} - \frac{i}{2}\Gamma_{H}  \end{matrix} \right) 
\end{align}
where $Q$ is a matrix composed of vertically stacked $v_1$ and $v_2$, and $P$ is an auxiliary matrix introduced to simplify the notation. 
Using mentioned notation the time evolution of the flavor states can be written as:

\begin{align}
\label{eq:time_evaluation}
  \left(\begin{matrix} \ket{M(t)}  \\ \ket{\overline{M}(t)} \end{matrix}  \right) 
  = QPQ^{-1} \left(\begin{matrix} \ket{M}  \\ \ket{M} \end{matrix}  \right) 
  = \left(\begin{matrix} g_{+}(t) &  \frac{q}{p}g_{-}(t)  \\  \frac{p}{q}g_{-}(t)  & g_{+}(t)  \end{matrix}\right) 
  \left(\begin{matrix} \ket{M}  \\ \ket{M} \end{matrix}  \right)   
\end{align}

where the parameters $g_{\pm}$ are given as:

\begin{align}
    g_{+} &= e^{-imt}e^{\Gamma \frac{t}{2}} \left[ \cosh(\frac{\Delta \Gamma t}{4})\cos(\frac{\Delta M t}{2})  - i\sinh(\frac{\Delta \Gamma t}{4})\sin(\frac{\Delta M t}{2})  \right] \\ 
    g_{-} &= e^{-imt}e^{\Gamma \frac{t}{2}} \left[ \sinh(\frac{\Delta \Gamma t}{4})\cos(\frac{\Delta M t}{2})  - i\cosh(\frac{\Delta \Gamma t}{4})\sin(\frac{\Delta M t}{2})  \right]   
\end{align}
and $m = \frac{1}{2}(M_{L}+M_H)$, $\Gamma =  \frac{1}{2}(\Gamma_{L}+\Gamma_{H})$, $\Delta M = M_{H}- M_{L}$, $ \Delta \Gamma =(\Gamma_{L}-\Gamma_{H})$.

The equation \ref{eq:time_evaluation} allows expressing the probability of mixing from particle to its antiparticle and vise-versa. Those probabilities are given by 

\begin{align}
\label{eq:mixing_prob}
|\bra{\overline{M}} \mathcal{H} \ket{M}|^{2} = \left|\frac{q}{p}\right|^{2} |g_{-}(t)| = \frac{e^{\Gamma t}}{2} \left|\frac{q}{p}\right|^{2}  \left[ \cosh(\frac{\Delta \Gamma t}{2})\cos(\Delta M t) \right] \\ 
|\bra{M} \mathcal{H} \ket{\overline{M}}|^{2} = \left|\frac{p}{q}\right|^{2} |g_{-}(t)| = \frac{e^{\Gamma t}}{2} \left|\frac{p}{q}\right|^{2}  \left[ \cosh(\frac{\Delta \Gamma t}{2})\cos(\Delta M t) \right]
\end{align}

The equation \ref{eq:mixing_prob} clearly indicates that the mass difference between light and heavy states drives the mixing frequency. It is also important to analyze ratio $\frac{p}{q}$. If it differs from 1 then the mixing process violate $CP$ symmetry. The word average of this mixing parameter for $B^{0}$ is  $\frac{p}{q} = 1.0009 \pm 0.0013$ and  for $B_s$  $\frac{p}{q} = 1.0003 \pm 0.0014$ \cite{PDG}, which means the results are consistent with conservation of $CP$ symmetry. 

\section{Baryon Asymmetry of the Universe and Sakharov conditions}

The previous sections described the combined $CP$ symmetry and discussed possible channels to study its violation. Here comes a vital question. Why do the researchers study the $CP$ violation? Why is it so crucial that even it is mentioned inside of the logo of the LHCb experiment? 

One of the answers to this question is the problem of the asymmetry of baryons with respect to anti-baryons. The current observed Universe is filled with baryons.  This observation is contradicted to the cosmological measurement, which strongly indicates that in the era of the early Universe, the matter and antimatter should have been created in equal amounts. Therefore, there must have been some process that had created the matter-antimatter asymmetry.  
In 1967 Russian physicist Sakharov postulated three conditions that have to be fulfilled to make baryogenesis, or in other words existence of known Universe,  possible \cite{sakharov}: 

\begin{enumerate}
    \item \textbf{Baryon number violation}.  All known perturbative processes in the Standard Model result in equal numbers of baryons and anti-baryons. However, there are non-perturbative electroweak processes that can produce baryons without anti baryons \cite{bayron_number_violation}. 
    \item \textbf{$C$ and $CP$ violation}. Violation of these symmetries are required even if there are processes, see the first condition, that could generate more baryon that anti-baryon an opposing process would generate an excess of anti-baryons, so the net baryon number of the system would still remain the same.
    \item \textbf{Departure from thermal equilibrium}. Baryogenesis cannot occur at thermal equilibrium; otherwise, the inverse of this process will occur at the same rate, and a net asymmetry will not be generated.
\end{enumerate}

The first condition is fulfilled by one of the default property of the Standard Model that requires the baryon minus lepton number to be conserved. Still, it does not have any restriction on the conservation of each of these numbers individually. The mechanism that would allow satisfying the third condition is the electroweak phase transition. Although, due to the mass of the Higgs boson $m_H \approx 126 GeV/c^{2}$, the phase transition would only be weakly first order and not provide a strong enough departure from thermal equilibrium \cite{phase_transiton}. 

 Within the Standard Model framework, the only source of the violation of $CP$ symmetry is the weak interactions in the quark sector. Although the known $CP$ violation in the quark sector is orders of magnitude too small to explain the baryon - anti-baryon asymmetry in the Universe, and therefore it is likely that this additional CP-violation originates in physics beyond the Standard Model. Therefore, precise comparisons of CP-violating observables and the Standard Model's predictions provide an invaluable probe of the New Physics.